%*******************************************************************************
%*********************************** First Chapter *****************************
%*******************************************************************************

%********************************** %First Section  **************************************
\chapter{Overview}
\section{A Brief Introduction to Stock Options} %Section - 1.1 

In finance, an option is a contract which gives the buyer (the owner or holder of the option) the right, but not the obligation, to buy or sell an underlying asset or instrument at a specified strike price on a specified date, depending on the form of the option. (Wikipedia). For the intent of this paper, an underlying asset or instrument will be considered to be a simple equity (or "stock", "share", etc.) in a firm. 

There are two types of options that will be priced as part of this simulation. The first, the \textit{Call} option, allows an investor to secure purchase of an asset at a certain prespecified price. The second, the \textit{Put} option, allows an investor to secure sale of an asset at a certain prespecified price. These prespecified prices are referred to as the 'strike' prices of the option. An option is considered worth exercising (buying or selling) in circumstances where it is \textit{in the money}. An option is in-the-money for a call when the strike price is below the underlying asset's value. It is in-the-money for a put when the strike price is above the market price of the asset. Typically, in options trading, a call option is used when investor sentiment on a stock is positive, while a put option is used when sentiment on a stock is negative. 

The style of option being bought or sold adds additional complexity to the situation. Each stock option being bought or sold is given an expiry date. An \textit{expiry date} is the date at which an option expires and a final decision must be made as to whether to exercise. There are two commonly traded types of options, however- American-Style and European-Style. American-Style stock options may be exercised at any point prior to the expiry. European-Style stock options may only be exercised at the time of expiry. This paper will solely explore American-Style options pricing as it relates to actual US market values.

\section{Simulation of Options}

With the advent of the modern financial services industry, high frequency trading, and computational data analytics it is now common practice to determine via computer simulation the estimated future prices of stock options. Numerous methods have been developed which address the idea of option estimation, the most notable being the \textbf{Black-Scholes} (sometimes Black-Scholes-Merton) method. The Black-Scholes method, introduced in 1983, is largely considered the standard for options pricing. A deep discussion of Black-Scholes is beyond the scope of this paper, but there are several points worth covering. 

The Black-Scholes model makes several key assumptions. The first is that there are two possible asset categories, the stock itself and an alternate lower-risk asset. The second is that the return of the stock can be estimated by a random walk with drift. This random walk is based on geometric Brownian motion, and is most simply a random path up and down the value of an asset might take. The random walk for equities is dictated by volatility, a concept which will be discussed shortly. 

It's also worth noting the original design of the Black-Scholes pricing model focused on pricing of European-Style options, but future refinements and iterations provided additional methods for the pricing of American-Style options. Two of these will be explored in the simulation, the Barone-Adesi and Whaley method and the Bjerksund-Stensland method. Both of these are approximation methods.

In order to model an option price, the Black-Scholes model requires several set variables. These are as follows:

\begin{description}
\item[Current underlying price:] The current market price of the equity.
\item[Option strike price:] The price at which the shares will be bought or sold.
\item[Time to expiry:] Time left until the option expires.
\item[Implied Volatility:] The size of the shifts the asset may make during the random walk.
\item[Risk-free interest rate:] The rate the asset would appreciate at in a lower-risk investment.
\end{description}

Another commonly used model for estimating stock option prices is the \textbf{binomial model}. A binomial model formulates a random option pricing curve based on a number of time intervals. The stock price is simulated to move up or down at each step by an amount relating to volatility and time to expiry. It is somewhat analogous to the Black-Scholes model, but is a discrete-time model as opposed to a continuous one. The result of binomial method options calculations resembles a tree, as the decision is made to exercise or hold at each point.

Luckily, the binomial method requires exactly the same inputs as the Black-Scholes model. As such, we can luckily maintain a similar testing framework for both. We will explore the following binomial methods as part of this simulation:

\begin{itemize}
\item{Cox-Ross-Rubenstein}
\item{Jarrow-Rudd}
\item{Equal Probabilities}
\item{Trigeorgis}
\item{Tian}
\item{Leisen-Reimer}
\end{itemize}

Binomial methods are less commonly used in professional options simulation\ref{???}. However, their ability to progressively price-in the value of an early option exercise makes it ideal for American options. 

\subsection{Volatility}
The asset's implied volatility is worth some discussion. Because volatility dictates the shifts of the random walk in the model, it has the potential to dictate the accuracy of the model altogether. There is some discussion as to whether historical volatility should be used in lieu of implied volatility. One of the simplest methods for calculating implied volatility is a guess-and-check approach involving repeated simulation volatility values. These values are calibrated until a result similar to the current option price is attained\ref{??}. This is less than ideal for our analysis, as it is not quantifiable. It also does not take into account whether current options are reasonably priced or are outliers.
\subsubsection{The GARCH Model}
A model commonly used in academia for quantitative finance volatility calculations is the GARCH model. The GARCH model is an autoregressive conditional heteroskedasticity method that is used to model time series\ref{}. In this case, we specifically will be using the GARCH(1,1) model for calculations. 

The GARCH model produces an $\alpha$ and $\beta$ value, as well as an $\omega$ value which represent the time series of stock prices. It assumes that the volatility value is mean reverting, or that it will always stay close to a specified value. Typically, $\alpha$ and $\beta$ are less than 1, which indicates that we can calculate the volatility as the standard deviation of the unconditional variance. This is represented by

\begin{equation}
\sigma^2 = \frac{\omega}{1 - \alpha - \beta}
\end{equation}

As is commonly known, the variance can be reduced down to the standard deviation by taking its square root.

\section{Intent of Analysis}

The intent of our analysis on these topics was to try and answer several key questions on the models:

\begin{description}
\item[Black-Scholes vs. Binomial] Which options pricing class provides better accuracy?
\item[Short term accuracy] Which options pricing model is the most accurate in the short term?
\item[Long term accuracy] Which options pricing model is the most accurate in the long term?
\item[Patterns] What patterns do we see in the accuracies/inaccuracies of these models?
\end{description}

Our analyses of these questions will be based on several datasets. The first will be a short-dated expiry set of puts and calls expiring one calendar week from the test. The second will be a long-dated expiry set of puts and calls expiring one 30-day month from the test. The third test will be the longest-dated expiry set of puts and calls expiring six 30-day months from the test. This will provide an effective cross-section of estimation methods over both the short and long-term time periods. 


