\chapter{Conclusions}

We used several accuracy methods to determine out if the Black-Scholes method was better than the binomial method, including the sum squared error, average sum squared error, and threshold accuracy calculations. After viewing the results of the accuracy methods between the two Black-Scholes and six binomial models to estimate the value of stock option prices, we found that there are very little differences between the sum squared error, average sum squared error, and  threshold accuracy calculations of each stock option price estimator model. Therefore, there seems to be no significant difference in the effectiveness of one model over another.  

After we performed the error and accuracy calculations on our datasets, we decided to plot the actual prices versus the estimated model's results to visualize the data. We plotted one week, six month, and eighteen month periods of puts and calls of an Apple, Inc. stock using the Cox-Ross-Rubenstein estimator model. We found that for all time periods, the put and call estimations tended to be overestimates compared to the actual prices. For the longer time periods of six months and eighteen months, the put estimations seemed to be more accurate than the call estimations as the put prediction values were closer to the actual prices. We also noticed that options with strike prices closer to the current price of the share are much harder to predict. During these tests, Apple, Inc. shares were priced at about \$98. In each chart, the estimator seemed to have the largest error in price for both put and call estimators at a strike price of \$98.

After viewing the results from the one week, six month, and eighteen month periods, we see that we can confidently conclude that there are very little differences between the sum squared errors between the eight pricing models. The sum squared errors always seem to remain relatively close to each other for each of the datasets. In our tests, no one pricing model consistently performed better than others when it came to estimating real prices. However, we did generally observe that the sum squared error was higher with call option estimations. Based on this pattern of higher sum squared error for the call options and lower threshold accuracy for the call options, we can confidently say that the eight pricing models make better estimations for put options than for call options. While we could not confidently determine if any one pricing model is superior to another, we can say that the models do seem to be reasonably accurate.
